\documentclass{article}

\usepackage{../zwliew}

\title{MA2101 - Linear Algebra II}
\author{Liew Zhao Wei}
\date{Semester 1, 2023-2024}

\begin{document}
\maketitle
\hrule

\section{Vector spaces}

\subsection{Subspaces}

\begin{definition}[Subspace]
  A subset $U$ of $V$ is a \emph{subspace} of $V$ if it is also a vector space using the same operations as $V$.
\end{definition}
It is troublesome to prove all the axioms of a vector space for a subset.
It turns out that we only need to prove that the following conditions hold.

\begin{theorem}[Conditions for a subspace]
  $U$ is a subspace of $V$ if and only if the following conditions hold:
  \begin{enumerate}
    \item $0 \in U$ (i.e. additive identity exists)
    \item If $u, v \in U$, then $u + v \in U$. (i.e. closed under addition)
    \item If $u \in U$, then $c \in \field$, $cu \in U$. (i.e. closed under scalar multiplication)
  \end{enumerate}
\end{theorem}

The notion of a direct sum (and hence, a sum) is important in the study of linear maps.

\begin{definition}[Sum of subsets]
  Let $U_1, \ldots, U_m$ be subsets of $V$.
  The sum of $U_1, \ldots, U_m$, denoted $U_1 + \cdots + U_m$, is defined as
  \[
    U_1 + \cdots + U_m = \setpred{u_1 + \cdots + u_m}{u_i \in U_i}
  \]
  That is, it is the set of all possible sums of elements from $U_1, \ldots, U_m$.
\end{definition}

\begin{theorem}[Sum of subspaces is smallest containing subspace]
  Let $U_1, \ldots, U_m$ be subspaces of $V$.
  Then, $U_1 + \cdots + U_m$ is the smallest subspace of $V$ containing $U_1, \ldots, U_m$.
\end{theorem}

\begin{definition}[Direct sum]
  Let $U_1, \ldots, U_m$ be subspaces of $V$.
  The sum $U_1 + \cdots + U_m$ is a \emph{direct sum} if each element of $U_1 + \cdots + U_m$ can be written in exactly one way as $u_1 + \cdots + u_m$ with $u_i \in U_i$.
  The direct sum is denoted as $U_1 \oplus \cdots \oplus U_m$.
\end{definition}
It is clear that sums of subspaces are analogous to unions of subsets, while direct sums of subspaces are analogous to disjoint unions of subsets.

\begin{theorem}[Condition for a direct sum]
  Let $U_1, \ldots, U_m$ be subspaces of $V$.
  Then, $U_1 + \cdots + U_m$ is a direct sum if and only if the only solution to $u_1 + \cdots + u_m = 0$ with $u_i \in U_i$ is $u_1 = \cdots = u_m = 0$.
\end{theorem}
With this, there is a simple condition for the sum of two subspaces to be a direct sum.

\begin{theorem}[Direct sum of two subspaces]
  Let $U$ and $W$ be subspaces of $V$.
  Then, $U + W$ is a direct sum if and only if $U \cap W = \set{0}$.
\end{theorem}

\section{Linear maps}

\subsection{Isomorphism and invertibility}

\begin{theorem}[Inverse and invertible]
  A linear map $S \in \mathcal{L}(W, V)$ is an \emph{inverse} of a linear map $T \in \mathcal{L}(V, W)$ if $ST = I_V$ and $TS = I_W$.
  $T$ is said to be \emph{invertible} if it has an inverse.
\end{theorem}

\begin{remark}[Inverses are unique]
  An invertible linear map has a unique inverse.
\end{remark}

\begin{theorem}[Invertiblity is equivalent to bijectivity]
  A linear map is invertible if and only if it is bijective.
\end{theorem}

\begin{definition}[Isomorphism]
  An \emph{isomorphism} is an invertible linear map.
  Two vector spaces are \emph{isomorphic} if there is an isomorphism from one to the other.
\end{definition}

\begin{theorem}[Isomorphic vector spaces have the same dimension]
  Two finite-dimensional vector spaces are isomorphic if and only if they have the same dimension.
\end{theorem}

\begin{theorem}[$\mathcal{L}(V, W)$ is isomorphic to $M_{m \times n}(\field)$]
  Let $V$ and $W$ be finite-dimensional vector spaces with $\dim V = n$ and $\dim W = m$.
  Then, $\mathcal{M}$ is an isomorphism between $\mathcal{L}(V, W)$ and $\field_{m \times n}$.
\end{theorem}

\begin{corollary}[$\dim \mathcal{L}(V, W) = (\dim V)(\dim W)$]
  Let $V$ and $W$ be finite-dimensional vector spaces.
  Then, $\dim \mathcal{L}(V, W) = (\dim V)(\dim W)$.
\end{corollary}

\begin{definition}[Linear operator, $\mathcal{L}(V)$]
  A linear map from a vector space $V$ to itself is called a \emph{linear operator} and is denoted by $\mathcal{L}(V)$.
\end{definition}

\begin{theorem}[Injectivity is equivalent to surjectivity]
  Let $T \in \mathcal{L}(V)$ be a linear operator on a finite-dimensional vector space $V$.
  Then, $T$ is injective if and only if it is surjective.
\end{theorem}

\subsection{Duality}

\begin{definition}[Linear functional]
  A \emph{linear functional} $T \in \mathcal{L}(V, \field)$ is a linear map from a vector space to its underlying field.
\end{definition}

\begin{example}[Examples of linear functionals]
  We present some common examples of linear functionals.
  \begin{enumerate}
    \item Let $(c_1, \ldots, c_n) \in \field^n$.
          Then, the map $f(x_1, \ldots, x_n) = c_1x_1 + \cdots + c_nx_n$ is a linear functional on $\field^n$.
    \item The \emph{trace} of an $n \times n$ matrix $A$, defined as $\tr A = \sum_{i = 1}^n a_{ii}$, is a linear functional on $\field_{n \times n}$.
    \item Let $[a, b]$ be a closed interval in $\reals$.
          Then, the map $L(f) = \int_a^b f(x) \,dx$ is a linear functional on the space of continuous real-valued functions on $[a, b]$, denoted $C([a, b])$.
  \end{enumerate}
\end{example}

\begin{definition}[Dual space]
  The \emph{dual space} of a vector space $V$, denoted $V^*$, is the vector space of all linear functionals on $V$.
  That is, $V^* = \mathcal{L}(V, \field)$.
\end{definition}

\begin{theorem}[$\dim V^* = \dim V$]
  Let $V$ be a finite-dimensional vector space.
  Then, $\dim V^* = \dim V$.
\end{theorem}

\subsection{Matrix transpose}

\begin{definition}[Transpose of a matrix, $A^t$]
  Let $A$ be an $m \times n$ matrix.
  The \emph{transpose} of $A$ is the $n \times m$ matrix $A^t$ defined by $A_{ij}^t = A_{ji}$.
  That is, the rows and columns of $A$ are interchanged.
\end{definition}

\begin{theorem}[Transpose of matrix products]
  Let $A$ be an $m \times n$ matrix and $B$ be an $n \times p$ matrix.
  Then, $(AB)^t = B^tA^t$.
\end{theorem}

(Skipping some definitions here regarding the dual basis and going straight to more pertinent results -- will revisit it later.)

\begin{definition}[Row rank, column rank]
  Let $A \in \field^{m \times n}$ be an $m \times n$ matrix.
  \begin{itemize}
    \item The \emph{row rank} of $A$ is the dimension of the span of the rows of $A$ in $\field^{1, n}$.
    \item The \emph{column rank} of $A$ is the dimension of the span of the columns of $A$ in $\field^{m, 1}$.
  \end{itemize}
\end{definition}

\begin{theorem}[$\dim R(T)$ equals to column rank of $\mathcal{M}(T)$]
  Let $V$ and $W$ be finite-dimensional vector spaces and $T \in \mathcal{L}(V, W)$.
  Then, $\dim R(T)$ equals the column rank of $\mathcal{M}(T)$.
\end{theorem}

\begin{theorem}[Row rank equals column rank]
  Let $A \in \field^{m \times n}$ be an $m \times n$ matrix.
  Then, the row rank of $A$ is equal to the column rank of $A$.
\end{theorem}

Since the row rank and column rank are equal, we can simply refer to the rank of a matrix.

\begin{definition}[Rank]
  The \emph{rank} of a matrix $A$, denoted $\rank A$, is the column rank of $A$.
\end{definition}

\section{Trace and Determinant}

Trace and determinants are primarily used to establish the properties of eigenvalues in the next section.

\begin{definition}[Characteristic polynomial]
  Let $T$ be an operator on a complex vector space.
  If $T$ has distinct eigenvalues $\lambda_1, \ldots, \lambda_m$ with multiplicities $d_1, \ldots, d_m$, then the \emph{characteristic polynomial} of $T$ is defined as
  \begin{align}
    p_T(z) = (z - \lambda_1)^{d_1} \cdots (z - \lambda_m)^{d_m}
  \end{align}
\end{definition}

\subsection{Trace}

\begin{definition}[Trace of an operator]
  The trace of an operator $T$, denoted $\tr T$, is defined as follows:
  \begin{itemize}
    \item If $\field = \complex$, then $\tr T$ is the sum of the eigenvalues of $T$, counted with multiplicity.
    \item If $\field = \reals$, then $\tr T$ is the sum of the eigenvalues of $T_\complex$, counted with multiplicity.
  \end{itemize}
\end{definition}

\begin{theorem}[Relation between trace to characteristic polynomial]
  For any operator $T \in \mathcal{L}(V)$ with $n = \dim V$, $\tr T$ is equal to the negative of the coefficient of $z^{n - 1}$ in the characteristic polynomial of $T$.
\end{theorem}

\begin{definition}[Trace of a matrix]
  The \emph{trace} of an $n \times n$ matrix $A$, denoted $\trace A$, is the sum of the diagonal entries of $A$.
  That is, $\tr A = \sum_{i = 1}^n a_{ii}$.
\end{definition}

We will show that the trace of an operator is equal to the trace of any of its matrix representation.

\begin{lemma}[Trace is cyclic]
  For any two square matrices $A$ and $B$ of the same size,
  \begin{align}
    \tr AB = \tr BA
  \end{align}
\end{lemma}

\begin{lemma}[Trace of matrix is oblivious to basis]
  Suppose $T$ is a linear operator.
  For any two bases $\mathcal{B}$ and $\mathcal{C}$ of a finite-dimensional vector space,
  \begin{align}
    \tr \mathcal{M}_\mathcal{B}(T) = \tr \mathcal{M}_\mathcal{C}(T)
  \end{align}
\end{lemma}

\begin{theorem}[Trace of linear operator and its matrix are equal]
  For any operator $T$,
  \begin{align}
    \trace T = \tr \mathcal{M}(T)
  \end{align}
\end{theorem}

\begin{corollary}[Trace is additive]
  For any two operators $S$ and $T$,
  \begin{align}
    \tr (S + T) = \tr S + \tr T
  \end{align}
\end{corollary}

\begin{theorem}[$ST - TS \neq I$]
  For any two operators $S$ and $T$,
  \begin{align}
    ST - TS \neq I
  \end{align}
\end{theorem}

\subsection{Determinant}

\begin{definition}[Determinant of an operator]
  The determinant of an operator $T$, denoted $\det T$, is defined as follows:
  \begin{itemize}
    \item If $\field = \complex$, then $\det T$ is the product of the eigenvalues of $T$, counted with multiplicity.
    \item If $\field = \reals$, then $\det T$ is the product of the eigenvalues of $T_\complex$, counted with multiplicity.
  \end{itemize}
\end{definition}

Similar to the trace, the determinant is closely related to the characteristic polynomial.

\begin{theorem}[Relation between determinant and characteristic polynomial]
  For any operator $T \in \mathcal{L}(V)$ with $n = \dim V$, $\det T$ is equal to $(-1)^n$ times the constant term of the characteristic polynomial of $T$.
\end{theorem}

\begin{remark}
  Combining the trace and determinant, we can rewrite the characteristic polynomial as
  \begin{align}
    p_T(t) = z^n - (\tr T)z^{n - 1} + \cdots + (-1)^n \det T
  \end{align}
\end{remark}

\begin{theorem}[Invertible iff nonzero determinant]
  An operator $T$ is invertible if and only if $\det T \neq 0$.
\end{theorem}

\begin{theorem}[Invertible iff nonzero determinant]
  A square matrix $A$ is invertible if and only if $\det A \neq 0$.
\end{theorem}

\section{Eigenvalues and Eigenvectors}

\begin{definition}[Invariant subspace]
  A subspace $U \subseteq V$ is \emph{invariant} under a linear operator $T$ if $T(U) \subseteq U$.
\end{definition}

\begin{example}[Examples of trivially invariant subspaces]
  For any linear operator $T \in \mathcal{L}(V)$, the following subspaces are invariant under $T$:
  \begin{itemize}
    \item $\set{0}$
    \item $V$
    \item $\ker T$
    \item $R(T)$
  \end{itemize}
\end{example}

Of particular interest are the invariant subspaces of dimension $1$.

\begin{definition}[Eigenvalue and eigenvector]
  A scalar $\lambda \in \field$ is an \emph{eigenvalue} of a linear operator $T \in \mathcal{L}(V)$ if there exists a nonzero vector $v \in V$ such that $Tv = \lambda v$.
  The vector $v$ is called an \emph{eigenvector} of $T$ corresponding to $\lambda$.
\end{definition}

\begin{remark}[Equivalent definition of eigenvector]
  A non-zero vector $v \in V$ is an eigenvector of $T$ corresponding to $\lambda$ if and only if $v \in \ker (T - \lambda I)$.
\end{remark}

With that remark, we arrive at the following equivalences for eigenvalues.

\begin{theorem}[Equivalent conditions for eigenvalues]
  Let $V$ be finite-dimensional, $T \in \mathcal{L}(V)$, and $\lambda \in \field$.
  Then, the following are equivalent:
  \begin{enumerate}
    \item $\lambda$ is an eigenvalue of $T$.
    \item $T - \lambda I$ is not injective.
    \item $T - \lambda I$ is not surjective.
    \item $T - \lambda I$ is not invertible.
  \end{enumerate}
\end{theorem}

\begin{theorem}[Eigenvectors corresponding to distinct eigenvalues are linearly independent]
  Let $T \in \mathcal{L}(V)$.
  Every list of eigenvectors of $T$ corresponding to distinct eigenvalues is linearly independent.
\end{theorem}

\begin{theorem}[Number of eigenvalues is at most $\dim V$]
  Any linear operator on a finite-dimensional vector space $V$ has at most $\dim V$ distinct eigenvalues.
\end{theorem}

\begin{definition}[$T^m$]
  Let $T \in \mathcal{L}(V)$ and $m \in \nats$.
  \begin{itemize}
    \item $T^0$ is defined as the identity operator $I$ on $V$.
    \item $T^m \in \mathcal{L}(V)$ is defined as $T^m = T \circ T^{m - 1}$.
    \item If $T$ is invertible, then $T^{-m} \in \mathcal{L}(V)$ is defined as $T^{-m} = (T^{-1})^m$.
  \end{itemize}
\end{definition}

\begin{remark}
  $T^m T^n = T^{m + n}$ and $(T^m)^n = T^{mn}$.
\end{remark}

\begin{definition}[Matrix similarity]
  Two order-$n$ square matrices $A$ and $B$ are \emph{similar} if there is an invertible matrix $P$ such that
  \begin{align}
    B = P^{-1}AP
  \end{align}
  Similar matrices represent the same linear map under two different bases.
\end{definition}

\begin{remark}[Similarity equivalence]
  Similarity is an equivalence relation on the set of all order-$n$ square matrices.
\end{remark}

\begin{definition}[Characteristic polynomial equals $\det (zI - T)$]
  The characteristic polynomial of an operator $T$ is equal to $\det (zI - T)$.
\end{definition}

\begin{remark}
  Similar matrices have the same characteristic polynomial.
\end{remark}

\begin{theorem}[Multiplicies sum to $\dim V$]
  For any operator $T$ on a complex vector space $V$, the sum of the multiplicities of the eigenvalues of $T$ is equal to $\dim V$.
\end{theorem}

\end{document}
