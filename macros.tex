%!TEX root=./main.tex

\theoremstyle{definition}
\newtheorem{definition}{Definition}
\newtheorem{theorem}{Theorem}
\theoremstyle{lemma}
\newtheorem{lemma}{Lemma}
\newtheorem{corollary}{Corollary}
\newtheorem{claim}{Claim}
\newtheorem*{remark}{Remark}
\theoremstyle{definition}
\newtheorem*{example}{Example}

% Labels and references
\newcommand{\figlabel}[1]{\label{fig:#1}}
\newcommand{\figref}[1]{Figure~\ref{fig:#1}}
\newcommand{\seclabel}[1]{\label{sec:#1}}
\newcommand{\secref}[1]{Section~\ref{sec:#1}}
\newcommand{\chaplabel}[1]{\label{chap:#1}}
\newcommand{\chapref}[1]{Chapter~\ref{chap:#1}}
\newcommand{\exlabel}[1]{\label{ex:#1}}
\newcommand{\exref}[1]{Example~\ref{ex:#1}}
\newcommand{\deflabel}[1]{\label{def:#1}}
\newcommand{\defref}[1]{Definition~\ref{def:#1}}
\newcommand{\problabel}[1]{\label{prob:#1}}
\newcommand{\probref}[1]{Problem~\ref{prob:#1}}
\newcommand{\thmlabel}[1]{\label{thm:#1}}
\newcommand{\thmref}[1]{Theorem~\ref{thm:#1}}
\newcommand{\proplabel}[1]{\label{prop:#1}}
\newcommand{\propref}[1]{Proposition~\ref{prop:#1}}
\newcommand{\lemlabel}[1]{\label{lem:#1}}
\newcommand{\lemref}[1]{Lemma~\ref{lem:#1}}
\newcommand{\corlabel}[1]{\label{cor:#1}}
\newcommand{\corref}[1]{Corollary~\ref{cor:#1}}
\newcommand{\itmlabel}[1]{\label{itm:#1}}
\newcommand{\itmref}[1]{\ref{itm:#1}}
\newcommand{\rulelabel}[1]{\label{rule:#1}}
\newcommand{\ruleref}[1]{Rule~\ref{rule:#1}}
\newcommand{\equlabel}[1]{\label{eq:#1}}
\newcommand{\equref}[1]{Equation~(\ref{eq:#1})}
\newcommand{\applabel}[1]{\label{app:#1}}
\newcommand{\appref}[1]{Appendix~\ref{app:#1}}
% \newcommand{\appref}[1]{ the Appendix}
\newcommand{\algolabel}[1]{\label{algo:#1}}
\newcommand{\algoref}[1]{Algorithm~\ref{algo:#1}}
\newcommand{\linlabel}[1]{\label{line:#1}}
\newcommand{\linref}[1]{Line~\ref{line:#1}}
\newcommand{\linrangeref}[2]{Lines~\ref{line:#1}-\ref{line:#2}}
\newcommand{\secrangeref}[2]{Sections~\ref{sec:#1}-\ref{sec:#2}}
\newcommand{\tablabel}[1]{\label{tab:#1}}
\newcommand{\tabref}[1]{Table~\ref{tab:#1}}
\newcommand{\remlabel}[1]{\label{rem:#1}}
\newcommand{\remref}[1]{Remark~\ref{rem:#1}}
\newcommand{\obslabel}[1]{\label{obs:#1}}
\newcommand{\obsref}[1]{Observation~\ref{obs:#1}}
\newcommand{\claimlabel}[1]{\label{claim:#1}}
\newcommand{\claimref}[1]{Claim~\ref{claim:#1}}

%% Colors
\definecolor{LightRed}{rgb}{1, 0.9, 0.9}
\definecolor{LightGreen}{rgb}{0.98, 1.0, 0.98}
\definecolor{DarkGreen}{rgb}{0.5, 0.7, 0.5}

%% Environments
\newcounter{questionCounter}
\setcounter{questionCounter}{0}
\newcommand{\question}{\addtocounter{questionCounter}{1}\fcolorbox{DarkGreen}{LightGreen}{\textbf{Question} \thequestionCounter.}}
\newcommand{\questionopt}{\addtocounter{questionCounter}{1}\fcolorbox{DarkGreen}{LightGreen}{\textbf{Question} \thequestionCounter. (Extra Credit)}}
\newcommand\hcancel[2][black]{\setbox0=\hbox{$#2$}%
\rlap{\raisebox{.45\ht0}{\textcolor{#1}{\rule{\wd0}{1pt}}}}#2} 

% \newenvironment{Solution}[3]{        % Riddle environment
%   \begin{center}                   % Centering minipage
%     \colorbox[HTML]{#1} {          % Set's the color of minipage
%       \begin{minipage}[b]{380px}   % Starts minipage
%    \textbf{#2}\\ \textit{#3}       % Set's title and starts italic for text
%   \end{minipage} }                 % End minipage
% }{\end{center}}

% \newenvironment{soln}
%   {\par\medskip
%    \color{red}%
%    \begin{framed}
%    \textbf{I say: }\ignorespaces}
%  {\end{framed}
%   \medskip}

\newenvironment{Solution}
  {
  % \par\medskip
   \color{red}%
   \begin{mdframed}[linecolor=OliveGreen,%
	backgroundcolor=Yellow!2]
   \noindent\textbf{Solution}.\\\\}
 {\end{mdframed}
  \medskip}
% \begin{mdframed}[backgroundcolor=blue!20] 
%         In any right triangle, the area of the square whose side is the hypotenuse is equal to the sum of the areas of the squares whose sides are the two legs.
%     \end{mdframed}

\newenvironment{Question}
{\begin{description}
	\item[\question]
}
{\end{description}
\medskip}

\newenvironment{QuestionOpt}
{\begin{description}
  \item[\questionopt]
}
{\end{description}
\medskip}



%% Math
\newcommand{\set}[1]{\{#1\}}
\newcommand{\setpred}[2]{\{#1 \,|\, #2\}}
\newcommand{\arity}[1]{\mathsf{arity}(#1)}
\newcommand{\powerset}[1]{\mathscr{P}(#1)}
\renewcommand{\emptyset}{\varnothing}

\newcommand{\nats}{\mathbb{N}}
\newcommand{\integers}{\mathbb{Z}}
\newcommand{\complexnum}{\mathbb{C}}
\newcommand{\reals}{\mathbb{R}}
\newcommand{\rationals}{\mathbb{Q}}

%% Logic
\newcommand{\props}{\euscr{P}}
\newcommand{\conn}{\euscr{C}}
\newcommand{\operators}{\euscr{O}}
\newcommand{\func}{\euscr{F}}
\newcommand{\reln}{\euscr{R}}
\newcommand{\vars}{\euscr{V}}
\newcommand{\struct}{\euscr{A}}
\newcommand{\fvar}[1]{\mathsf{FVar}(#1)}
\newcommand{\terms}{\mathsf{Terms}}
\newcommand{\val}[2]{\mathsf{val}_{#1}(#2)}
\newcommand{\wff}{\mathsf{FORM}}
\newcommand{\indwff}{\mathsf{INDFORM}}
\newcommand{\wffalt}{\mathsf{AltFORM}}
\newcommand{\indwffalt}{\mathsf{AltINDFORM}}
\newcommand{\truVal}{\texttt{true}}
\newcommand{\flsVal}{\texttt{false}}
\newcommand{\entails}{\vdash}


%% Graphs
\newcommand{\succr}[1]{\mathsf{succ}(#1)}
\newcommand{\pred}[1]{\mathsf{pred}(#1)}
\newcommand{\indeg}[1]{\mathsf{inDegree}(#1)}
\newcommand{\outdeg}[1]{\mathsf{outDegree}(#1)}
