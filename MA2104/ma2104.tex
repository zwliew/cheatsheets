\documentclass{article}

\usepackage{../zwliew}

\title{MA2104 - Multivariable Calculus}
\author{Liew Zhao Wei}
\date{Semester 1, 2023-2024}

\begin{document}
\maketitle
\hrule

\section{Vectors}

Multivariable calculus is the study of \emph{scalar fields} and \emph{vector fields}.

\begin{definition}[Scalar Field]
  A \emph{scalar field} is a map $f \colon X \subseteq \reals^n \to \reals$.
  It is also called a \emph{scalar-valued} function.
\end{definition}

\begin{definition}[Vector Field]
  A \emph{vector field} is a map $\bvec{f} \colon X \subseteq \reals^n \to \reals^m$.
  It is also called a \emph{vector-valued} function.
\end{definition}

\section{Coordinate Systems}

Cylindrical and spherical coordinates are both natural extensions of the polar coordinates in $\reals^2$, but in slightly different ways -- cylindrical coordinates tack on a $z$-coordinate, while spherical coordinates tack on an angle $\rho$ from the positive $z$-axis.

\subsection{Cylindrical Coordinates}

\emph{Cylindrical coordinates} are useful for describing objects with an axis of symmetry.

\begin{definition}[Cylindrical Coordinate Conversions]
  We can convert between cylindrical and cartesian coordinates like so:

  Cylindrical to Cartesian: $
    \begin{cases}
      x  = r\cos\theta
      \\
      y  = r\sin\theta
      \\
      z = z
    \end{cases}
  $
  Cartesian to Cylindrical: $
    \begin{cases}
      r^2 = x^2 + y^2
      \\
      \tan{\theta}  = y / x
      \\
      z = z
    \end{cases}
  $
\end{definition}
We typically impose the constraints $r \geq 0$ and $0 \leq \theta < 2\pi$ to ensure that all points on $\reals^3$, except those on the $z$-axis, are uniquely represented.

\subsection{Spherical Coordinates}

\emph{Spherical coordinates} are useful for describing objects with a centre of symmetry.

\begin{definition}[Spherical Coordinate Conversions]
  We can convert between sphericla and cartesian coordinates like so:

  Spherical to Cartesian: $
    \begin{cases}
      x  = \rho\sin\varphi\cos\theta
      \\
      y  = \rho\sin\varphi\sin\theta
      \\
      z = \rho\cos\varphi
    \end{cases}
  $
  Cartesian to Spherical: $
    \begin{cases}
      \rho^2 = x^2 + y^2 + z^2
      \\
      \tan{\theta}  = y / x
      \\
      \cos{\varphi} = \sqrt{x^2 + y^2} / z
    \end{cases}
  $
\end{definition}
We typically impose the constraints $\rho \geq 0$, $0 \leq \theta < 2\pi$ and $0 \leq \varphi \leq \pi$ to ensure that all points on $\reals^3$, except the origin, are uniquely represented.

\subsection{Hyperspherical Coordinates}

We can even generalize spherical coordinates to higher dimensions in the form of \emph{hyperspherical coordinates}.
The idea is to tack on more angles to the spherical coordinates, so that we can describe objects with more axes of symmetry.

\begin{definition}[Hyperspherical Coordinate Conversions]
  Spherical to Cartesian: $
    \begin{cases}
      x_1 = \rho \sin \varphi_1 \cdots \sin \varphi_{n-2} \cos \varphi_{n-1}
      \\
      x_2 = \rho \sin \varphi_1 \cdots \sin \varphi_{n-2} \sin \varphi_{n-1}
      \\
      x_3 = \rho \sin \varphi_1 \cdots \sin \varphi_{n-3} \cos \varphi_{n-2}
      \\
      x_4 = \rho \sin \varphi_1 \cdots \sin \varphi_{n-4} \cos \varphi_{n-3}
      \\
      \vdots
      \\
      x_n = \rho \cos \varphi_1
    \end{cases}
  $
\end{definition}
Notice the similarity with spherical coordinates.

Just like for spherical coordinates, we typically impose the constraints
\begin{align*}
  0 \leq  & \rho
  \\
  0  \leq & \varphi_i \leq \pi     & \text{ for $i = 1, \ldots, n - 2$}
  \\
  0  \leq & \varphi_{n - 1} < 2\pi
\end{align*}
to ensure that all points on $\reals^n$, except the origin, are uniquely represented.

\section{Basic Topology}

We begin with some basic topological definitions and results that we'll use later on.

\begin{definition}[Neighbourhood]
  A \emph{neighbourhood} of a point $p$ is a set $N_r(p)$ consisting of all points $q$ such that $\norm{q - p} < r$ for some $r > 0$.
\end{definition}
An \emph{open ball} with center $p \in \reals^n$ and radius $r$ is a neighbourhood $N_r(p)$ in $\reals^n$.
A \emph{closed ball} is similarly defined, but with $\norm{q - p} \leq r$ instead.

\begin{definition}[Limit point]
  A point $p$ is a \emph{limit point} of a set $S$ if every neighbourhood of $p$ contians a point $q \in X$ such that $q \neq p$.
  Such a point is also called an \emph{accumulation point}.
\end{definition}

\begin{definition}[Isolated point]
  An \emph{isolated point} of a set $S$ is a point $p \in S$ such that $p$ is not a limit point of $S$.
  That is, there is some neighbourhood of $p$ that contains no other points of $S$.
\end{definition}

\begin{definition}[Interior point]
  An \emph{interior point} of a set $S$ is a point $p \in S$ such that there is some neighbourhood of $p$ that is contained in $S$.
\end{definition}

\begin{definition}[Open set]
  A set $S$ is \emph{open} if every point of $S$ is an interior point of $S$.
\end{definition}

\begin{definition}[Closed set]
  A set $S$ is \emph{closed} if it contains all of its limit points.
\end{definition}

\begin{theorem}[Complement of open set is closed]
  A set is open if and only if its complement is closed.
\end{theorem}

\section{Limits and Continuity}

\begin{definition}[Limit]
  Consider a function $\bvec{f} \colon X \subseteq \reals^n \to \reals^m$.
  We write
  \begin{align}
    \lim_{\bvec{x} \to \bvec{a}} \bvec{f}(\bvec{x}) = \bvec{L}
  \end{align}
  to mean that given any $\epsilon > 0$, there is a $\delta > 0$ such that if $0 < \norm{\bvec{x} - \bvec{a}} < \delta$, then $\norm{\bvec{f}(\bvec{x}) - \bvec{L}} < \epsilon$.
\end{definition}
Intuitively, this means that $\bvec{f}(\bvec{x})$ can be made arbitrarily close to $\bvec{L}$ by taking $\bvec{x}$ sufficiently close to $\bvec{a}$.
The geometric interpretation is that given an open ball $B_\epsilon$ centred at $\bvec{L}$, there is an open ball $B_\delta$ centred at $\bvec{a}$ such that $\bvec{f}(\bvec{x})$ is contained in $B_\epsilon$ whenever $\bvec{x}$ is contained in $B_\delta$.

The usual theorems concerning limits of sums, products, and quotients hold for scalar fields.
For vector fields, we have natural extensions of these theorems, but without quotients.
\begin{theorem}[Algebraic Properties of Limits]
  Suppose that $\lim\limits_{\bvec{x} \to \bvec{a}} \bvec{f}(\bvec{x}) = \bvec{L}$ and $\lim\limits_{\bvec{x} \to \bvec{a}} \bvec{g}(\bvec{x}) = \bvec{M}$.
  Then
  \begin{enumerate}
    \item $\lim\limits_{\bvec{x} \to \bvec{a}} \bvec{f}(\bvec{x}) + \bvec{g}(\bvec{x}) = \bvec{L} + \bvec{M}$
    \item $\lim\limits_{\bvec{x} \to \bvec{a}} c\bvec{f}(\bvec{x}) = c\bvec{L}$ for any scalar $c$
    \item $\lim\limits_{\bvec{x} \to \bvec{a}} \bvec{f}(\bvec{x}) \cdot \bvec{g}(\bvec{x}) = \bvec{L} \cdot \bvec{M}$
    \item $\lim\limits_{\bvec{x} \to \bvec{a}} \norm{\bvec{f}(\bvec{x})} = \norm{\bvec{L}}$
  \end{enumerate}
\end{theorem}
$(3)$ is proved by rewriting
\begin{align*}
  \bvec{f}(\bvec{x}) \cdot \bvec{g}(\bvec{x}) - \bvec{L} \cdot \bvec{M} = (\bvec{f}(\bvec{x}) - \bvec{L}) \cdot (\bvec{g}(\bvec{x}) - \bvec{M}) + \bvec{L} \cdot (\bvec{g}(\bvec{x}) - \bvec{M}) + \bvec{M} \cdot (\bvec{f}(\bvec{x}) - \bvec{L})
\end{align*}
$(4)$ is proved by using $(3)$ and the fact that $\norm{\bvec{v}} = \sqrt{\bvec{v} \cdot \bvec{v}}$.

\begin{theorem}[Uniqueness of Limits]
  If $\lim_{\bvec{x} \to \bvec{a}} \bvec{f}(\bvec{x})$ exists, then it is unique.
\end{theorem}

\begin{theorem}[Limit Exists Iff Limit of Components Exist]
  Consider a function $\bvec{f} \colon X \subseteq \reals^n \to \reals^m$.
  Then $\lim\limits_{\bvec{x} \to \bvec{a}} \bvec{f}(\bvec{x})$ exists if and only if $\lim_{\bvec{x} \to \bvec{a}} f_i(\bvec{x})$ exists for each $i = 1, \ldots, m$.
  In this case,
  \begin{align}
    \lim_{\bvec{x} \to \bvec{a}} \bvec{f}(\bvec{x}) = \left( \lim_{\bvec{x} \to \bvec{a}} f_1(\bvec{x}), \ldots, \lim_{\bvec{x} \to \bvec{a}} f_m(\bvec{x}) \right)
  \end{align}
\end{theorem}
This is proved by observing that $f_i = \bvec{f} \cdot \bvec{e}_i$, and applying the algebraic properties of limits.

\begin{definition}[Continuity]
  Consider a function $\bvec{f} \colon X \subseteq \reals^n \to \reals^m$ and let $a \in X$.
  $\bvec{f}$ is \emph{continuous at $\bvec{a}$} if either $\bvec{a}$ is an isolated point of $X$ or if
  \begin{align}
    \lim_{\bvec{x} \to \bvec{a}} \bvec{f}(\bvec{x}) = \bvec{f}(\bvec{a})
  \end{align}
  If $\bvec{f}$ is continuous at every point of $X$, then we simly say that $\bvec{f}$ is \emph{continuous}.
\end{definition}
Observe that many of the properties of limits also hold for continuity.

\begin{theorem}[Continuity of Composite Functions]
  If $\bvec{f} \colon X \subseteq \reals^n \to \reals^m$ is continuous at $\bvec{a}$ and $\bvec{g} \colon Y \subseteq \reals^m \to \reals^p$ is continuous at $\bvec{f}(\bvec{a})$, then $\bvec{g} \circ \bvec{f}$ is continuous at $\bvec{a}$.
\end{theorem}

\section{Derivative}

\begin{definition}[Directional Derivative]
  Consider a scalar field $f \colon X \subseteq \reals^n \to \reals$ and let $\bvec{a} \in X$.
  If $\bvec{v}$ is a unit vector, then the \emph{directional derivative of $f$ at $\bvec{a}$ in the direction of $\bvec{v}$} is
  \begin{align}
    D_{\bvec{v}}f(\bvec{a}) = \lim_{h \to 0} \frac{f(\bvec{a} + h\bvec{v}) - f(\bvec{a})}{h}
  \end{align}
  if the limit exists.
\end{definition}

\begin{definition}[Partial Derivative]
  Consider a scalar field $f \colon X \subseteq \reals^n \to \reals$ and let $\bvec{a} \in X$.
  The \emph{partial derivative of $f$ with respect to $x_i$ at $\bvec{a}$} is
  the directional derivative of $f$ at $\bvec{a}$ in the direction of $\bvec{e}_i$.
  It is also denoted by $f_{x_i}(\bvec{a})$ or $D_{x_i}f(\bvec{a})$.
\end{definition}

The definitions of differentiability for scalar fields generalize naturally to vector fields, so I will simply state the ones for vector fields.

\begin{definition}[Jacobian Matrix]
  Consider a vector field $\bvec{f} \colon X \subseteq \reals^n \to \reals^m$.
  Define $D\bvec{f}$ to be
  \begin{align}
    D\bvec{f} = \begin{bmatrix}
                  \dfrac{\partial f_1}{\partial x_1} & \cdots & \dfrac{\partial f_1}{\partial x_n}
                  \\
                  \vdots                             & \ddots & \vdots
                  \\
                  \dfrac{\partial f_m}{\partial x_1} & \cdots & \dfrac{\partial f_m}{\partial x_n}
                \end{bmatrix}
  \end{align}
  $D\bvec{f}$ is called the \emph{Jacobian matrix} of $\bvec{f}$.
  The $i$th row of $D\bvec{f}$ is the gradient vector $\nabla f_i$.
\end{definition}
Observe that $D\bvec{f}$ is a linear map from $\reals^n$ to $\reals^m$.
We will see that if a derivative exists at $\bvec{a}$, then it must be equal to $D\bvec{f}(\bvec{a})$.

\begin{definition}[Differentiability]
  Consider a vector field $\bvec{f} \colon X \subseteq \reals^n \to \reals^m$ and let $\bvec{a} \in X$.
  $\bvec{f}$ is \emph{differentiable at $\bvec{a}$} if the linear map $D\bvec{f}(\bvec{a})$ exists and
  \begin{align}
    \lim_{\bvec{x} \to \bvec{a}} \frac{\norm{\bvec{f}(\bvec{x}) - (\bvec{f}(\bvec{a}) + D\bvec{f}(\bvec{a})(\bvec{x} - \bvec{a}))}}{\norm{\bvec{x} - \bvec{a}}} = 0
  \end{align}
  $\bvec{f}$ is also said to be \emph{differentiable at $\bvec{a}$}.
\end{definition}
If $\bvec{f}$ is differentiable at $\bvec{a}$, then the function $\bvec{h}(\bvec{x}) = \bvec{f}(\bvec{a}) + D\bvec{f}(\bvec{a})(\bvec{x} - \bvec{a})$ is a good linear approximation of $\bvec{f}$ near $\bvec{a}$.

The results for scalar fields generalize naturally to vector fields too.

\begin{theorem}[Differentiability Implies Continuity]
  If $\bvec{f} \colon X \subseteq \reals^n \to \reals^m$ is differentiable at $\bvec{a}$, then it is also continuous at $\bvec{a}$.
\end{theorem}

\begin{theorem}[Continuity and Existence of Partial Derivatives Imply Differentiability]
  If $\bvec{f} \colon X \subseteq \reals^n \to \reals^m$ is continuous at $\bvec{a}$ and if all the partial derivatives of $\bvec{f}$ exist in some neighbourhood of $\bvec{a}$, then $\bvec{f}$ is differentiable at $\bvec{a}$.
\end{theorem}

\begin{theorem}[Differentiable Iff Components are Differentiable]
  Consider a vector field $\bvec{f} \colon X \subseteq \reals^n \to \reals^m$.
  $\bvec{f}$ is differentiable at $\bvec{a}$ if and only if each component function $f_i$ is differentiable at $\bvec{a}$.
\end{theorem}

\end{document}
