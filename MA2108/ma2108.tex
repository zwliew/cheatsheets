\documentclass{article}

\usepackage{../zwliew}

\title{MA2108 - Mathematical Analysis I}
\author{Liew Zhao Wei}
\date{Semester 1, 2023-2024}

\begin{document}
\maketitle
\hrule

\section{Preface}

Here are just some quick scribbles of theorems that are used in the course.
More complete notes will be written in the future.

\section{Real Numbers}

\begin{theorem}[Triangle Inequality]
  For any real numbers $a$ and $b$, we have
  \begin{align}
    \abs{\abs{a} - \abs{b}} \leq \abs{a \pm b} \leq \abs{a} + \abs{b}
  \end{align}
\end{theorem}

\section{Sequences}

\begin{theorem}[Convergence Implie Boundedness]
  Every convergent sequence is bounded.
\end{theorem}

\begin{theorem}[Algebraic Properties of Limits]
  Limits preserve addition, subtraction, multiplication, and division.
\end{theorem}

We will later see that limits also preserve square roots and absolute values.

\begin{theorem}[Squeeze Theorem]
  Suppose $x_n$ and $y_n$ converge to $a$.
  If there is a natural number $K$ such that $x_n \leq y_n \leq z_n$ for all natural numbers $n \geq K$, then $y_n \to a$.
\end{theorem}

\begin{theorem}
  If $\abs{x_n} \to 0$, then $x_n \to 0$.
\end{theorem}

\begin{theorem}[Bernoulli's Inequality]
  For every real number $r \geq 1$ and $x \geq -1$, $(1 + x)^r \geq 1 + rx$.
\end{theorem}

\begin{theorem}
  For any \textbf{fixed} $-1 < b < 1$, $b^n \to 0$.
\end{theorem}

\begin{example}
  $\dfrac{1}{2^n} \to 0$ and $\left(\dfrac{2}{3}\right)^n \to 0$
\end{example}

\begin{theorem}
  For a fixed $c > 0$, $c^{1/n} \to 1$.
\end{theorem}

\begin{theorem}
  If $x_n \to x$, then $\abs{x_n} \to \abs{x}$.
\end{theorem}

\begin{theorem}
  If $x_n \to x$, then $\sqrt{x_n} \to \sqrt{x}$.
\end{theorem}

\begin{theorem}
  $n^{1/n} \to 1$.
\end{theorem}

\begin{theorem}[Limits Preserve Inequalities]
  Suppose $x_n \to x$ and $y_n \to y$.
  \item If $x_n \geq y_n$ for all $n \in \nats$, then $x \geq y$.
\end{theorem}

\begin{corollary}
  By Theorem 3.12,
  \begin{enumerate}
    \item If $x_n \geq 0$ for all $n \in \nats$, then $\lim\limits_{n \to \infty} x_n \geq 0$.
    \item If $a, b \in \reals$ and $a \leq x_n \leq b$ for all $n \in \nats$, then $a \leq \lim\limits_{n \to \infty} x_n \leq b$.
  \end{enumerate}
\end{corollary}

\begin{theorem}
  $\left( 1 + \frac{1}{n} \right)^n \to e$
\end{theorem}

\begin{theorem}[Monotone Convergence Theorem]
  If $(x_n)$ is monotone and bounded, then $x_n$ converges to:
  \begin{enumerate}
    \item $\sup \set{x_n}$ if $(x_n)$ is increasing
    \item $\inf \set{x_n}$ if $(x_n)$ is decreasing
  \end{enumerate}
\end{theorem}

\begin{remark}
  If $(y_k) = (x_{n_k})$ is a subsequence of $(x_n)$, then $n_k \geq k$.
\end{remark}

\begin{theorem}[Sequence Converges Implies Subsequences Converge]
  If $(x_n)$ converges to $x$, then any subsequence $(x_{n_k})$ also converges to $x$.
\end{theorem}

\begin{theorem}[Monotone Subsequence Theorem]
  Every sequence has a monotone subsequence.
\end{theorem}

\begin{theorem}[Bolzano-Weierstrass Theorem]
  Every bounded sequence has a convergent subsequence.
\end{theorem}

\subsection{Cauchy Convergence Criterion}

\begin{definition}[Cauchy Sequence]
  A sequence $\set{x_n}$ is \emph{Cauchy} if for every $\epsilon > 0$, there is an integer $N$ such that $\abs{s_n - s_m} < \epsilon$ whenever $n, k \geq N$.
\end{definition}

\begin{example}
  The sequence $x_n = \dfrac{1}{n}$ is Cauchy, but $x_n = (-1)^n$ is not.
\end{example}

\begin{theorem}[Cauchy Implies Boundedness]
  Every Cauchy sequence is bounded.
\end{theorem}

\begin{theorem}[Cauchy and Convergent Subsequence Implies Convergence]
  If $\set{x_n}$ is Cauchy and has a convergent subsequence, then $\set{x_n}$ converges.
\end{theorem}

\begin{theorem}[Cauchy Convergence Criterion]
  A sequence $\set{x_n}$ in $\reals^k$ is Cauchy if and only if it converges.
\end{theorem}

\begin{definition}[Contractive Sequence]
  A sequence $\set{x_n}$ is \emph{contractive} if there is a constant $0 \leq C < 1$ such that $\abs{x_{n + 2} - x_{n + 1}} \leq C \abs{x_{n + 1} - x_{n}}$ for all $n \in \nats$.
\end{definition}

\begin{theorem}[Contractive Sequences are Cauchy]
  Every contractive sequence is Cauchy.
\end{theorem}

\begin{theorem}[Increasing and Unbounded Sequences are Properly Divergent]
  If a sequence $\set{x_n}$ is increasing and not bounded above, then $x_n \to \infty$.
\end{theorem}

\begin{definition}[Properly Divergent Sequence]
  A sequence $\set{x_n}$ is \emph{properly divergent} if either $x_n \to \infty$ or $x_n \to -\infty$.
\end{definition}

\begin{theorem}
  If $k < \ell$ and $1 < a < b$, then $n^k << n^\ell << a^n << b^n << n!$.
\end{theorem}

\section{Series}

Analysis was mainly motivated by the study of infinite series, which are sums of infinitely many terms.

\begin{definition}[Convergence of Series]
  A series $\sum_{n=1}^\infty x_n$ \emph{converges} if the sequence of partial sums $\set{s_n}$ converges.
\end{definition}

\begin{example}
  $\sum\limits_{i = 1}^\infty 2^{-\abs{i}}$ converges to $3$.
  $\sum\limits_{n = 1}^\infty \dfrac{1}{n(n + 1)}$ converges to $1$.
\end{example}

\begin{theorem}[Convergence of Geometric Series]
  If $\abs{r} < 1$, then $\sum\limits_{n = 0}^\infty r^n$ converges to $\dfrac{1}{1 - r}$.
\end{theorem}

\begin{theorem}
  Let $\set{x_n}$ be a sequence and let $M \in \nats$.
  $\sum\limits_{n = 1}^\infty x_n$ converges if and only if $\sum\limits_{n = M}^\infty x_n$ converges.
\end{theorem}

\begin{definition}[Cauchy Series]
  A series $\sum x_n$ is \emph{Cauchy} if the sequence of partial sums $\set{s_n}$ is Cauchy.
\end{definition}

\begin{theorem}[Cauchy Convergence Criterion for Series]
  A series $\sum x_n$ converges if and only if it is Cauchy.
\end{theorem}

\begin{theorem}
  A series $\sum x_n$ is Cauchy if and only if for every $\epsilon > 0$, there is an integer $N$ such that $\abs{\sum\limits_{k = n + 1}^m x_k} < \epsilon$ whenever $n, m \geq N$.
\end{theorem}

\begin{theorem}[$n$th Term Test]
  If a series $\sum x_n$ converges, then $\lim\limits_{n \to \infty} x_n = 0$.
\end{theorem}

The converse of the $n$th term test is not true.

\begin{theorem}[Divergence of Harmonic Series]
  The \emph{harmonic series} $\sum\limits_{n = 1}^\infty \dfrac{1}{n}$ diverges.
\end{theorem}

\begin{theorem}
  If $\abs{r} \geq 1$, then $\sum\limits_{n = 0}^\infty r^n$ diverges.
\end{theorem}

\begin{corollary}
  The series $\sum\limits{n = 0}^\infty \alpha(r)^n$ converges if and only if $\abs{r} < 1$.
\end{corollary}

\begin{theorem}[Comparison Test]
  Suppose that there is a natural number $K \in \nats$ such that $0 \leq x_n \leq y_n$ for all $n \geq K$.
  Then,
  \begin{enumerate}
    \item If $\sum y_n$ converges, then so does $\sum x_n$.
    \item If $\sum x_n$ diverges, then so does $\sum y_n$.
  \end{enumerate}
\end{theorem}

\begin{theorem}[Convergence Criterion for $p$-series]
  The $p$-series $\sum\limits_{n = 1}^\infty \dfrac{1}{n^p}$ converges if and only if $p > 1$.
\end{theorem}

\begin{definition}[Eventually Non-negative Series]
  A series $\sum x_n$ is \emph{eventually non-negative} if there is a natural number $N$ such that $x_n \geq 0$ for all $n \geq N$.
\end{definition}

\begin{theorem}
  If $\sum x_n$ is eventually non-negative, then it converges if and only if $\set{s_n}$ is bounded above.
\end{theorem}

\begin{definition}[Eventually Positive Series]
  A series $\sum x_n$ is \emph{eventually positive} if there is a natural number $N$ such that $x_n > 0$ for all $n \geq N$.
\end{definition}

\begin{theorem}[Limit Comparison Test]
  Consider two eventually positive series $\sum x_n$ and $\sum y_n$ where the limit
  \begin{align}
    \rho = \lim_{n \to \infty} \frac{x_n}{y_n}
  \end{align}
  exists.
  Then,
  \begin{enumerate}
    \item If $\rho > 0$, then the two series either both converge or both diverge.
    \item If $\rho = 0$ and $\sum y_n$ converges, then $\sum x_n$ converges.
  \end{enumerate}
\end{theorem}

\begin{theorem}[Ratio Test]
  Consider an eventually positive series $\sum a_n$ where the limit
  \begin{align}
    \rho = \lim_{n \to \infty} \frac{a_{n + 1}}{a_n}
  \end{align}
  exists.
  Then,
  \begin{enumerate}
    \item If $\rho < 1$, then $\sum a_n$ converges.
    \item If $\rho > 1$, then $\sum a_n$ diverges.
    \item Otherwise, the test is inconclusive.
  \end{enumerate}
\end{theorem}

\end{document}
